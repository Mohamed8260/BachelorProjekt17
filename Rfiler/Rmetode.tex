\chapter{Metode}

\section{Tekniske udviklingsværktøjer}
\begin{itemize}
\item Analyse og designmetode\\
ASE-modellen
V-modellen

Dette afsnit har til formål at beskrive hvilke metoder, der er benyttet af udarbejdelsen af bachelorprojektet. Primært er der tale om metoder fra faget ISE. I dette afsnit bliver der også beskrevet hvilke arbejdsredskaber, der er benyttet til udførelse af bachelorprojektet og rapporten.\\

Til beskrivelse samt opbygning af Synkerefleksmonitor er der fra ISE benyttet metoden SysML. SysML er brugt til diagramanalyse, specifikation, design og verificerer Synkerefleksmonitor. Hvilket resulterer i en beskrivelse af systemets opbygning og kommunikation. Dernæst er der lavet en applikationsmodel, som giver det samlet overblik over Synkerefleksmonitor. Applikationsmodellen består af en domænemodel, hvor alle aktiviterne i synkerefleksmonitor er beskrevet samt tilhørende klassediagrammer med metoder fra sekvensdiagrammer som beskriver systemets virkning og interaktionen mellem de forskellige dele, som er specifikt for hvert use case.\\

Programmet Visio er blevet brugt til udvikling af alle SysML- og UML-diagrammer. Koden og GUI er skrevet og udviklet i Matlab. Rapporten, bilag, mødereferater og logbog er skrevet i tekstsproget Latex på hjemmesiden Overleaf. 

\section{Processen}

Bachelorprojektet startede med at få lavet en tidsplan over hele forløbet, med udkast fra bachelorforprojektet. Her blev der der brugt TeamGantt som projektstyringsværktøj, til oprettelse af tidsplanen, som er en online portal hvor alle gruppedeltager har mulighed for at se og rette i tidsplanen. Siden er bygget op om et Gantt-skeam som viser aktivterne i kalenderformat, som bruges til at dokumenterer planlægningen\cite[s. 297]{IntroductionCompendium}.Ved brug af versionshistorik af tidsplanen, var det muligt at følge ændringer undervejs i projektet. \textit{Se Bilag 2 for versionshistorik af tidsplanen}. Projektet brugte TeamGantt kun til grovplaner med strukturen efter ASE-modellen. Udførelsen af de enkelte elementer fra ASE-modellen blev udført ved brug af V-modellen, for at opretholde en høj kvalitet i projektet. V-modellen sikre at hver fase er færdig og giver mulighed for at test løbende før næste fase begynder\cite[s. 12]{DevelopmentASE}. 




Undervejs er de specifikke opgaver oprettet, for hvert sprint, i programmet Pivotal Tracker. Når en opgave blev oprettet blev der taget op i gruppen hvilken prioritering opgaven skulle have ved brug af en terning fra 1 til 8 point. Hvert medlem viste sine valgte point. Ved uoverensstemmelse af point skulle hvert medlem argumentere og der blev diskuteret i gruppen om en fælles prioritering af opgaven.  







\item SysML og UML
\item Husk referncer til litteratur, samt afvigelser fra teoriens metoder
\end{itemize}

\section{Beskriv Processen 1-2 sider - ikke tekniske del}
\begin{itemize}
\item Gruppedannelser\\
Bachelorprojekt gruppen 




\item Anvendelse af samarbejdsaftaler\\
Der er udarbejdet en samarbjdsaftale som kan ses i Bilag XX. Den er udarbejdet på bag grund af erfaringer fra tideligere projekter og hvad dette projekt kræver.
\item Arbejdsfordeling\\
Arbejdsfordelingen af de praktiskeopgaver og ansvarlige områder er beskrevet i samarbejdsaftalen. se bilag XX. Alle opgaver har været lagt ind i Pivotal Tracker hvor hvert medlem har kunne vælge opgaver efter interesse og efter den overordnet planlægning.
\item Planlægning\\
Teamgantt er den overordnet planlægning som bliver diskuteret og redigeret hver fredag efter et sprint er fuldendt.  
\item Møder\\
Scrum møder hver dag kl. 8:30 som omhandler igangværende opgaver og status og fremgangen på disse. Møde hver fredag. 14 omhandlende ugens sprint og status sprintes opgaver. 
\item Projektledelse\\
Der er en ligefordelt ledelse i gruppen, med et fælles ansvar, med roller, opgave planlægning og organisering. Se samarbejdskontrakt bilag XX.

\item Projektsadministation\\
Der er opdelt ansvarsopgaver i gruppen i mellem såsom, dokument ansvar og referant ved møder. Se den nærmere oversigt i samarbejdsaftalen Bilag XX.
\item Sprints (Scrum)\\
Gruppen har valgt at køre med ugentlige scrum sprints med værktøjet Pivotal Tracker. 
\end{itemize}
Den gennemførte proces beskrives nærmere i procesbeskrivelsen i projektets bilag.


