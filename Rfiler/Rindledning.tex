\chapter{Indledning}

\section{Baggrund}
En normal synkeproces er kendetegnet ved at fødeindtagelse som passerer fra den bageste del af mundhulen, via. svælget og spiserøret til mavesækken sker uden besvær. Tilstanden hvori selve synkefunktionen og dens hastighed og frekvens forstyrres kaldes for dysfagi [1]. Dysfagi er den medicinske betegnelse for symptom relateret til synkebesvær. Der er vigtigt at differentiere mellem nedre og øvre dysfagi. Øvre dysfagi omfatter præ-oral, oral og faryngeal fase, hvorimod nedre dysfagi relaterer sig den øsefageale fase dvs. mavesæk og spiserør [2]. Det skal dog nævnes, at der er uenigheder om definitionen af dysfagi. Den manglende konsensus om definitionen gør rapportering af dysfagi insidens og prævalens uklar. Ifølge patientombuddets temarapport fra 2012 om dysfagi fremgår det at:
\section{Problemformulering}